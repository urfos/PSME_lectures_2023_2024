\documentclass{beamer}

\usepackage{subfigure}
\usepackage{graphicx}
\usepackage{sidecap}
\usepackage{caption}
%\usepackage{subcaption}
\captionsetup{compatibility=false}
\usepackage{appendixnumberbeamer}
\usepackage{amsmath}
% --
\usepackage{multirow}
\usepackage{xcolor}
\usepackage{setspace}
\usepackage{hyperref}
\usepackage{anyfontsize}

\beamertemplatenavigationsymbolsempty
\setbeamertemplate{footline}

\AtBeginSection[]
{
    \begin{frame}
        \frametitle{Table of Contents}
        \tableofcontents[currentsection]
    \end{frame}
}

\AtBeginSubsection[]
{
    \begin{frame}
        \frametitle{Table of Contents}
        \tableofcontents[currentsection,currentsubsection]
    \end{frame}
}

\newenvironment{itemise} {\begin{itemize} \setlength{\itemsep}{0.2cm}} {\end{itemize}}
\usepackage[labelformat=empty]{caption}
\setbeamertemplate{sections/subsections in toc}[square]

%% COLORS
\definecolor{Gray}{gray}{0.9}
\definecolor{dblue}{rgb}{0.132,0.1,0.27}
\definecolor{mint}{cmyk}{1.0, 0.2, 0.6, 0.05}
\definecolor{ant}{cmyk}{0.5, 0.1, 0.0, 0.45}
\definecolor{lgray}{cmyk}{0.12, 0.0, 0.0, 0.17}
\definecolor{lred}{cmyk}{0.0, 0.9, 0.7, 0.0}


\usepackage{etoolbox}% http://ctan.org/pkg/etoolbox 
\usepackage{booktabs}

\newenvironment{literatur}{%
  \parskip2pt \parindent0pt \raggedright
  \def\lititem{\hangindent=0.5cm \hangafter1}}{%
  \par\ignorespaces}

\newcommand{\tb}[1]{{\color{blue}{\textbf{#1}}}}
\newcommand{\tm}[1]{{\color{mint}{\textbf{#1}}}}
\newcommand{\tr}[1]{{\color{red}{\textbf{#1}}}}
\newcommand{\trs}[1]{{\color{red}{#1}}}
\newcommand{\tms}[1]{{\color{mint}{#1}}}
\newcommand{\tps}[1]{{\color{purple}{#1}}}

% Ilya: packages

\usepackage{tikz}
\usepackage{lmodern}
\usepackage{enumitem}

% Ilya: my commands

\newenvironment{mytemize}
{\vfill\itemize[nolistsep,itemsep=\fill,label=\color{blue}{$\triangleright$}]}
  {\enditemize}


\newenvironment{mynumerate}
{\vfill\enumerate[nolistsep,itemsep=\fill,label=\arabic*.]}
  {\endenumerate}

\newcommand{\hitem}[1]{
  {\color{blue}{$\triangleright$}} 
  {#1} 
  {\hfill}
}

\setlist[itemize]{label= \color{blue}{$\triangleright$}}
\setlist[enumerate]{label = \arabic*.}

\newcommand{\rarr}{$\Rightarrow$\ }



%\href{<Ziel>}{<Eingefasster Text>} 

%\logo{\includegraphics[height=0.7cm]{BdFlogo.eps}\hspace{300pt}\vspace{-5pt}}
%\logo{\includegraphics[height=0.8cm]{BdFlogo.eps}}
%\logo{\pgfputat{\pgfxy(-6.2,-0.5)}{\pgfbox[center,base]{\includegraphics[height=0.8cm]{BdFlogo.eps}}}}

%------------------------------------------------------------------------------------
% TITLE
%------------------------------------------------------------------------------------
\title[PSME]{Macroeconomics\\ Lecture 5 -- Permanent Income} 
\author[I. Eryzhenskiy]{Ilya Eryzhenskiy}
\institute[BdF]{PSME Panth\'{e}on-Sorbonne Master in Economics}
\date[PSME macro]{Fall 2023}

%---BEGIN------------------------------------------------------------------------------
\begin{document}

\begin{frame}
  \maketitle
\end{frame}

\begin{frame}{Overview}
  \tableofcontents
\end{frame}


%---FRAME------------------------------------------------------------------------------
\section{Multi-period consumer problem}
%---FRAME------------------------------------------------------------------------------
\begin{frame}{Multi-period budget constraints}

The consumer lives from $t=0$ till $t=T$ ($T+1$ periods) \\ \vfill
Initial condition: savings $S_{-1}$ with interest $r_{-1}$:

\begin{equation*}
\begin{cases}
\begin{aligned}
C_{0}+S_{0}&=Y_{0}+\left( 1+r_{-1}\right) S_{-1}\\
C_{1}+S_{1}&=Y_{1}+\left( 1+r_{0}\right) S_{0}\\
&\dots \\
C_{t} + S_{t} &= Y_{t} + (1+r_{t-1}) S_{t-1} \\
C_{t+1}+S_{t+1}&=Y_{t+1}+\left( 1+r_{t}\right) S_{t}\\ 
&\dots \\
C_{T}+S_{T}&=Y_{T}+( 1+r_{T-1}) S_{T-1}
\end{aligned}
\end{cases}
\end{equation*}
$T \to \infty$ can also be considered \rarr \tb{infinite-horizon} model
\end{frame}


\begin{frame}{Inter-temporal budget constraint: step by step}
\begin{mynumerate}
    \item Solve for $S_0$ in period \tr{1} budget constraint:
	\begin{equation*}
	C_1 + S_{1} = (1+r_0) S_0 + Y_{1} - C_{1} \Leftrightarrow 
	S_{0} = \frac{C_{1} - Y_{1} - S_{1}}{1+r_0}
    \end{equation*}
 \item Plug the obtained expression of $S_0$ in the period \tr{0} budget constraint. $S_0$ is eliminated, but now $S_1$ in period 0 constraint 
 \item Do step 1 for $S_1$ in period \tr{2} constraint
 \item Plug obtained expression of $S_1$ back into equation of step 2. $S_1$ eliminated, but now $S_2$ in the equation
 \item Repeat for $S_{2}, S_{3} \dots S_{T}$
\end{mynumerate}

	
\end{frame}

\begin{frame}{Inter-temporal budget constraint}
    Denote cumulative returns as follows: $R_{0,0} = 1,\ R_{0,1} = (1+r_0), R_{0,2} = (1+r_0)(1+r_1)$ and $R_{0,t} = (1+r_0)(1+r_{1}) ... (1+r_{t-1})$\\
\vfill 
Then, the inter-temporal budget constraint with $T$ periods is:
$$\sum_{t=0}^{T} \frac{C_t}{R_{0, t}} = (1+r_{-1})S_{-1} + \sum_{t=0}^{T} \frac{Y_t}{R_{0, t}} + \underbrace{\frac{S_T}{R_{0, T}}}_{=0}$$
If we assume constant $r_t = r$, then $R_{0,t} = (1+r)^{t}$ and we obtain:

$$\sum_{t=0}^{T} \frac{C_t}{(1+r)^{t}} = (1+r)S_{-1} + \sum_{t=0}^{T} \frac{Y_t}{(1+r)^{t}} + \underbrace{\frac{S_T}{(1+r)^{T}}}_{=0}$$

    For $T\to \infty$, sufficient to assume that $S$ does not grow/fall at a rate faster than $r$: $\lim_{T\to \infty}{\frac{S_T}{(1+r)^{T}}}=0$ -- \tb{transversality condition} 
\end{frame}

\begin{frame}{Lifetime utility}
Each next period's instanteneous utility gets multiplied by $\beta<1$ one more time: \tm{geometric discounting}
\begin{align*}
U(C_0, C_{1},\dots C_{T}) &= u(C_0) + \beta u(C_{1}) +...+ \beta^{t} u(C_{t}) + ... + \beta^{T} u(C_{T}) \\
&= \sum_{t=0}^{T} \beta^t u(C_t)
\end{align*}
\end{frame}

\begin{frame}{A dynamic Lagrangian}
It is (paradoxically) more convenient to write a Lagrangian with $T+1$ period constraints instead of the inter-temporal one \rarr $T+1$ Lagrange multipliers 
$ \lambda_0, ..., \lambda_{t}, ..., \lambda_{T}$:
$$\mathcal{L} = \sum_{t=0}^{T} \beta^{t} [u(C_{t}) + \lambda_{t} (Y_{t}+(1+r_{t-1})S_{t-1}-C_{t}-S_{t})] $$
\begin{mytemize}
    \item discounting with $\beta^t$ applies to \textbf{both} period $t$ utility and period $t$ Lagrange multiplier (it doesn't mean anything, done for convenience)
    \item to obtain the Euler equation, derivative w.r.t. $C_t$ and either $C_{t+1}$ \textbf{or} $S_t$ \textcolor{mint}{($S_t$ is harder, but useful for future models)}
\end{mytemize}
\end{frame}

\begin{frame}{Euler equation as derivative with respect to $S_t$}
   Write periods $t$ and $t+1$ of the Lagrangian explicitly to see that $S_t$ appears twice:  \begin{align*}
\mathcal{L} =& 
%\sum_{k=0}^{t-1}\beta^k &[u(C_k) + \lambda_{k} (Y_{k}+(1+r_{k-1})S_{k-1}-C_{k}-S_{k})] \\
  u(C_0) + \lambda_{0} (Y_{0}+(1+r_{-1})S_{-1}-C_{0}-S_{0})  \\
  &+ \dots \\
  &+\beta^t [u(C_t) + \lambda_{t} (Y_{t}+(1+r_{t-1})S_{t-1}-C_{t}-\textcolor{red}{S_{t}})] \\
  &+\beta^{t+1} [u(C_{t+1}) + \lambda_{t+1} (Y_{t+1}+(1+r_{t})\textcolor{red}{S_{t}}-C_{t+1}-S_{t+1})] \\
   &+ \dots \\
  &+\beta^T [u(C_T) + \lambda_{T} (Y_{T}+(1+r_{T-1})S_{T-1}-C_{T}-S_{T})] 
\end{align*}
\end{frame}

\section{Permanent Income Hypothesis}

\subsection{A simplified model}
\begin{frame}{Permanent income hypothesis: simplifying assumptions}
   To simplify the study of temporary and permanent income shocks, make two assumptions: 
   \begin{enumerate}
       \item Constant real interest $r_t = r$
       \item $\beta = \frac{1}{1+r}$
   \end{enumerate}
   \vfill 
   Then, a constant consumption level $C_t = \bar C$ follows from Euler equation: 
   \begin{align*}
       \frac{u'(C_t)}{\beta u'(C_{t+1})} = 1+r &\Leftrightarrow u'(C_t) = \beta (1+r) u'(C_{t+1})  \\
       &\Leftrightarrow u'(C_t) = \frac{1}{1+r} (1+r) u'(C_{t+1}) \\ &\Leftrightarrow u'(C_t) = u'(C_{t+1}) \Leftrightarrow \textcolor{red}{C_t = C_{t+1} = \bar C}
   \end{align*}

This is an extreme case of \tb{consumption smoothing}
\end{frame}

\begin{frame}{Permanent consumption: calculation}
\vspace{-5cm}
   Substitute constant consumption level in intertemporal budget constraint, then use sum of geometric series: 
\end{frame}

\begin{frame}{Permanent consumption and income shocks: temporary}
   $$\bar C = \frac{1-\beta}{1-\beta^{T+1}}\left[(1+r_{-1})S_{-1} + \sum_{t=0}^T \beta^t Y_t\right]$$ 
   where $\frac{1-\beta}{1-\beta^{T+1}}<1$ \tm{(check this using $\beta^{T+1}<\beta$)} \\
   \vfill
Derivative with respect to income in $t=0$: $\frac{\partial \bar C}{\partial Y_0} = \frac{1-\beta}{1-\beta^{T+1}}$ \\
This is the \tb{mpc} in the Keynesian sense: share of increase in \textbf{current} income that is consumed. \\ 
\vfill
The mpc is smaller when the lifespan $T$ is larger. \\ If $T \to \infty,$ mpc $= 1-\beta$ \\
\vfill
Derivative with respect to income in any period $t$: $\frac{\partial \bar C}{\partial Y_t} = \beta^t \frac{1-\beta}{1-\beta^{T+1}} $ \\
The \textbf{further} in the future the income shock, the \textbf{smaller the reaction} of permanent consumption 
   
\end{frame}

\begin{frame}{Permanent consumption and income shocks: permanent}
   $$\bar C = \frac{1-\beta}{1-\beta^{T+1}}\left[(1+r)S_{-1} + \sum_{t=0}^T \beta^t Y_t\right]$$ 
   \vfill
   Take a total differential (with $d S_{-1} = 0$): 
   $$d \bar C = \frac{1-\beta}{1-\beta^{T+1}}\sum_{t=0}^T \beta^t d Y_t$$ 
Now consider a \tb{permanent shock}: $ d Y_0 = ... = d Y_t = ... = d Y_T:$
 $$d \bar C = \frac{1-\beta}{1-\beta^{T+1}} \cdot d Y_t  \cdot \sum_{t=0}^T \beta^t= d Y_t \cdot \underbrace{\frac{1-\beta}{1-\beta^{T+1}} \cdot \frac{1-\beta^{T+1}}{1-\beta}}_{=1}$$
We obtain $d \bar C = d Y_t$, \textbf{a one-to-one reaction of consumption to a permanent change in income} 
\end{frame}

\begin{frame}{Savings response to temporary and permanent shocks}
    What happens to inital period savings $S_0$ under different income shocks? \\
    \vfill Use formula $S_0 = (1+r)S_{-1} + Y_0 - C_0$:

    \begin{mynumerate}
        \item Temporary shock of current income $Y_0 \uparrow$:  \\ $C_0 \uparrow$, but $d Y_0 > d C_0$ because mpc $< 1$ \rarr \tr{$S_0 \uparrow$}
        \item Temporary shock of future income $Y_t$: \\ only $C_0 \uparrow$ in period 0 \rarr \tr{$S_0 \downarrow$}
        \item Permanent income shock: \\ one-to-one change in $C_0$ and $Y_0$ \rarr \tr{$S_0$ unchanged} 
    \end{mynumerate}
\end{frame}

\subsection{Uncertainty}
\begin{frame}{Uncertain income}

   Suppose now that income $\{Y_t\}_{t=0}^T$ is a random variable%, for example  an \tb{independently, identically distributed (i.i.d.)} normal variable: $Y_t \sim \matcal N (\bar Y, \sigma^2)$ \\
   \vfill
   Consumer then maximizes \tb{expected utility} \\
   \vfill
   $E_t X_k$ is expectation of period $k$ variable under information available in $t$ \\ \vfill The expected utility in period $0$ is $\tps{E_0} \sum_{t=0}^T \beta^t u(C_t)$
%   \begin{align*}
%       \max\ &E_0 \sum_{t=0}^T \beta^t u(C_t) \\
%       \text{s.t.}\ &C_t + S_t = Y_t + (1+r_{t-1}S_{t-1})    
%   \end{align*}
\vfill
   The Lagrangian then also has an expectation:
$$\mathcal{L} = \tps{E_0}\ \sum_{t=0}^{T} \beta^{t} [u(C_{t}) + \lambda_{t} (Y_{t}+(1+r_{t-1})S_{t-1}-C_{t}-S_{t})] $$
We will write the FOC for $S_t$ under \textbf{period $t$ information}:
$$\lambda_t = \beta \tps{E_t} [(1+r_t) \lambda_{t+1}] \ \Leftrightarrow \
u'(C_t) = \beta \tps{E_t} [(1+r_t)u'(C_{t+1})]$$
\end{frame}

\begin{frame}{Permanent income under uncertainty}
    Assume again $r_t = r$ and $\beta = \frac{1}{1+r}$ \\ \vfill
    Then, the Euler equation with information of period $t$ becomes: $$u'(C_t) = E_t u'(C_{t+1})$$
    If we also assume  a quadratic $u$, then we get $C_t = E_t C_{t+1}$ \rarr consumer \textbf{expects} to have constant consumption; consumption can only change due to \tb{unexpected shocks}
\end{frame}

\section{Multi-period small open economy}

  \begin{frame}{Multi-period IIP dynamics}
  A small open economy exists for $T+1$ periods $t=0, 1 ... T$.  \\ $IIP_t$ is \textbf{IIP of beginning of period $t$}. World interest rate is constant at $r$. $IIP$ dynamics depends on $TB$ as before:
	  \begin{align*}
   IIP_1 &= (1+r) IIP_0 + TB_0 \\
   &... \\
	  IIP_{t+1} &= (1+r) IIP_t + TB_t \\
	  IIP_{t+2} &= (1+r) IIP_{t+1} + TB_{t+1} \\
   &... \\
   IIP_T &= (1+r) IIP_{T-1} + TB_{T-1}
	  \end{align*}
   Country cannot have debts in last period ($IIP_{T} \geq 0$) nor hold assets in other countries ($IIP_{T} \leq 0$), so $IIP_{T} = 0$ 
  \end{frame}
  
  \begin{frame}{IPP and future trade balances step by step}
	In order to obtain the relationship of initial IIP and all the trade balances, same approach as with inter-temporal budget constraint: 
\begin{mynumerate}
    \item Solve for $IIP_1$ in equation with $TB_1$:
	\begin{equation*}
	IIP_2 = (1+r) IIP_1 + TB_1 \Leftrightarrow 
	IIP_{1} = \frac{-TB_1 + IIP_{2}}{1+r}
    \end{equation*}
 \item Plug the obtained expression of $IIP_1$ in the equation with $TB_0$. $IIP_1$ is eliminated, but now $IIP_2$ in the equation
 \item Do step 1 for $IIP_2$ in equation with $TB_2$
 \item Plug obtained expression of $IIP_2$ back into initial equation. $IIP_2$ eliminated, but now $IIP_3$ in the equation
 \item Repeat for $IIP_{3}, IPP_{4}, ... IIP_{T}$
\end{mynumerate}
\end{frame}

\begin{frame}{Multi-period economy: trade balance and solvency}
$$IIP_0 = \sum_{t=0}^T\frac{-TB_t}{(1+r)^{t+1}} + \underbrace{\frac{IIP_{T}}{(1+r)^{T+1}}}_{=0}$$
According to the initial value of International Investment Position, several scenarios for values of trade balance:
\begin{mytemize}
    \item If $IIP_t=0$, then trade deficit in one period must be compensated by a surplus in \textbf{at least one} other period
    \item If $IIP_{t} > 0$, it is possible to have a trade deficit in each period and remain solvent
    \item If $IIP_{t} < 0$, the economy may have to run trade surpluses every period to remain solvent. The earlier the surplus, the bigger role it has for debt repayment (interest accumulation)
\end{mytemize}
\vfill
Transversality condition for $T \to \infty$: $\lim_{T\to \infty}\frac{IIP_T}{(1+r)^{T+1}}=0$
\end{frame}

\begin{frame}{Summary}
    \begin{mytemize}
        \item We studied microeconomics of consumption choice and dynamics of small open economies
        \item A consumer has forward-looking behaviour that leads to mpc$<1$ and reaction to future income shocks
        \item Under specific assumptions on interest rate and discount factor, consumption is constant and reacts one-to-one to permanent income shocks
        \item Under uncertainty, expected utility is maximized subject to current information. Under simplifying assumptions, consumption changes only after unexpected shocks
        \item A small open economy has international investment position that changes with current account flows. In a representative consumer economy, IIP has similar dynamics to consumer savings 
    \end{mytemize}
\end{frame}
\end{document}