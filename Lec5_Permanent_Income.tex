\documentclass{beamer}

\usepackage{subfigure}
\usepackage{graphicx}
\usepackage{sidecap}
\usepackage{caption}
%\usepackage{subcaption}
\captionsetup{compatibility=false}
\usepackage{appendixnumberbeamer}
\usepackage{amsmath}
% --
\usepackage{multirow}
\usepackage{xcolor}
\usepackage{setspace}
\usepackage{hyperref}
\usepackage{anyfontsize}

\beamertemplatenavigationsymbolsempty
\setbeamertemplate{footline}

\newenvironment{itemise} {\begin{itemize} \setlength{\itemsep}{0.2cm}} {\end{itemize}}
\usepackage[labelformat=empty]{caption}
\setbeamertemplate{sections/subsections in toc}[square]

%% COLORS
\definecolor{Gray}{gray}{0.9}
\definecolor{dblue}{rgb}{0.132,0.1,0.27}
\definecolor{mint}{cmyk}{1.0, 0.2, 0.6, 0.05}
\definecolor{ant}{cmyk}{0.5, 0.1, 0.0, 0.45}
\definecolor{lgray}{cmyk}{0.12, 0.0, 0.0, 0.17}
\definecolor{lred}{cmyk}{0.0, 0.9, 0.7, 0.0}


\usepackage{etoolbox}% http://ctan.org/pkg/etoolbox 
\usepackage{booktabs}

\newenvironment{literatur}{%
  \parskip2pt \parindent0pt \raggedright
  \def\lititem{\hangindent=0.5cm \hangafter1}}{%
  \par\ignorespaces}

\newcommand{\tb}[1]{{\color{blue}{\textbf{#1}}}}
\newcommand{\tm}[1]{{\color{mint}{\textbf{#1}}}}
\newcommand{\tr}[1]{{\color{red}{\textbf{#1}}}}
% Ilya: packages

\usepackage{tikz}
\usepackage{lmodern}
\usepackage{enumitem}

% Ilya: my commands

\newenvironment{mytemize}
{\vfill\itemize[nolistsep,itemsep=\fill,label=\color{blue}{$\triangleright$}]}
  {\enditemize}


\newenvironment{mynumerate}
{\vfill\enumerate[nolistsep,itemsep=\fill,label=\arabic*.]}
  {\endenumerate}

\newcommand{\hitem}[1]{
  {\color{blue}{$\triangleright$}} 
  {#1} 
  {\hfill}
}

\setlist[itemize]{label= \color{blue}{$\triangleright$}}
\setlist[enumerate]{label = \arabic*.}

\newcommand{\rarr}{$\Rightarrow$\ }



%\href{<Ziel>}{<Eingefasster Text>} 

%\logo{\includegraphics[height=0.7cm]{BdFlogo.eps}\hspace{300pt}\vspace{-5pt}}
%\logo{\includegraphics[height=0.8cm]{BdFlogo.eps}}
%\logo{\pgfputat{\pgfxy(-6.2,-0.5)}{\pgfbox[center,base]{\includegraphics[height=0.8cm]{BdFlogo.eps}}}}

%------------------------------------------------------------------------------------
% TITLE
%------------------------------------------------------------------------------------
\title[PSME]{Macroeconomics\\ Lecture 5 -- Permanent Income} 
\author[I. Eryzhenskiy]{Ilya Eryzhenskiy}
\institute[BdF]{PSME Panth\'{e}on-Sorbonne Master in Economics}
\date[PSME macro]{Fall 2023}

%---BEGIN------------------------------------------------------------------------------
\begin{document}

\begin{frame}
  \maketitle
\end{frame}

\begin{frame}{Overview}
  \tableofcontents
\end{frame}


%---FRAME------------------------------------------------------------------------------
\section{Multi-period Consumer Problem}
%---FRAME------------------------------------------------------------------------------
\begin{frame}
\frametitle{Outline}
\tableofcontents[currentsection]
\end{frame}
%---FRAME------------------------------------------------------------------------------
\begin{frame}{Multi-period budget constraints}

The consumer lives $T$ periods: from $t=1$ to $t=T$ \\ \vfill
Initial condition: savings $S_{0}$ with accrued interest $r_{0}$:

\begin{equation*}
\begin{cases}
\begin{aligned}
C_{1}+S_{1}&=Y_{1}+\left( 1+r_{0}\right) S_{0}\\
C_{2}+S_{2}&=Y_{2}+\left( 1+r_{1}\right) S_{1}\\
&\dots \\
C_{t+1}+S_{t+1}&=Y_{t+1}+\left( 1+r_{t}\right) S_{t}\\ 
C_{t+2} + S_{t+2} &= Y_{t+2} + (1+r_{t+1}) S_{t+1} \\
&\dots \\
C_{T}+S_{T}&=Y_{T}+( 1+r_{T-1}) S_{T-1}
\end{aligned}
\end{cases}
\end{equation*}
$T \to \infty$ can also be considered \rarr \tb{infinite-horizon} model
\end{frame}


\begin{frame}{Inter-temporal budget constraint: derivation}
\vspace{-3.5cm}
First, solve for $S_1$ in period $2$ budget constraint:
	\begin{equation*}
	C_2 + S_{2} = (1+r_1) S_1 + Y_{2} - C_{2} \Leftrightarrow 
	S_{1} = \frac{C_{2} - Y_{2} - S_{2}}{1+r_1}
\end{equation*}
Then, plug this expression of $S_1$ in the period $2$ budget constraint. Then, repeat for $S_{2}, S_{3} \dots S_{T}:$
	
\end{frame}

\begin{frame}{Inter-temporal budget constraint}
    The product of returns from period $1$ to period $t$, $(1+r_1)\cdot(1+r_{2})\cdot ... \cdot (1+r_{t})$, will be denoted $R_{1, t}$ \\
\vfill 
Then, the inter-temporal budget constraint with $T$ periods can be written:
$$\sum_{t=1}^{T} \frac{C_t}{R_{1, t}} = (1+r_{0})S_{0} + \sum_{t=1}^{T} \frac{Y_t}{R_{1, t}} + \underbrace{\frac{S_T}{R_{1, T}}}_{=0}$$
If we assume $r_t = r$, the expression becomes:

$$\sum_{t=1}^{T} \frac{C_t}{(1+r)^{t}} = (1+r_{0})S_{0} + \sum_{t=1}^{T} \frac{Y_t}{(1+r)^{t}} + \underbrace{\frac{S_T}{(1+r)^{T}}}_{=0}$$
\end{frame}

\begin{frame}{Lifetime utility}
Each next period's instanteneous utility gets multiplied by $\beta<1$ one more time: \tm{geometric discounting}
$$U(C_1, C_{2},\dots C_{T}) = u(C_1) + \beta u(C_{2}) + \beta^{t-1} u(C_{t}) + ... + \beta^{T-1} u(C_{T})$$
    
\end{frame}

\begin{frame}{A dynamic Lagrangian}
It is (paradoxically) more convenient to write a Lagrangian with $T$ period constraints instead of the inter-temporal one \rarr $T$ Lagrange multipliers 
$ \lambda_t, \lambda_{t+1}, \dots \lambda_{t+T}$:
$$\mathcal{L} = \sum_{t=1}^{T} \beta^{t-1} [u(C_{t}) + \lambda_{t} (Y_{t}+(1+r_{t-1})S_{t-1}-C_{t}-S_{t})] $$
\begin{mytemize}
    \item discounting with $\beta^t$ applies to utility \textbf{and to Lagrange multiplier}
    \item possible to take a derivative with respect to $S_{t}$ to obtain Euler equation
\end{mytemize}
\end{frame}

\begin{frame}{Euler equation as derivative with respect to $S$}
   Write two periods of the Lagrangian explicitly: 
\begin{align*}
\mathcal{L} = \beta^t u(C_t)
\sum_{s = t}^{t+T} \beta^s [u(C_{s}) + \lambda_{s} (Y_{s}+(1+r_{s})S_{s}-C_{s}-S_{s+1})] $$
\end{align*}
\end{frame}

  \begin{frame}{IIP and future trade balances}
	\vspace{-5cm}
	$$IIP_{t+1} = (1+r_t) IIP_t + TB_t $$
	Recursive relationship. Substitute $IPP_{t+1}$, then $IPP_{t+2}$ and so on. Result?
	

  \end{frame}
  \begin{frame}{Simplified dynamics of IPP}
	Simplified BOP identity is $CA_t = FA_t$ (no capital account, no errors and omissions)\\
	\vfill
	Replace in the IPP dynamics formula (without valuation changes): 
	  \begin{align*}
	  IIP_{t+1} &= IIP_t + CA_t \\
	  			&= IIP_t + TB_t + r_t IIP_t \\
	  &= (1+r_t) IIP_t + TB_t 
	  \end{align*}
  \end{frame}
  \begin{frame}{IIP and future trade balances: interpretation}
	\begin{mytemize}
	  \item Assume $IIP_t<0$ -- country is a \textbf{net debtor} (not sovereign debt, but all sectors)
	  \item Then, country must run future \tb{trade surplus} on average (with more importance to surpluses that are in near future)
		\begin{mytemize}
		\item Otherwise, debts cannot be repaid
		\end{mytemize}
	  \item Assume $IIP_t>0$ -- country is a \textbf{net creditor} of rest of the world 
	  \item Then, country must run future \tb{trade deficits} on average (with more importance to deficits that are in near future)
		\begin{mytemize}
		\item Otherwise, debts of non-residents cannot be repaid
		\end{mytemize}
	\end{mytemize}
	\vfill
	\rarr Perpetual trade surpluses / trade deficits can be sustainable if very negative/very positive IIP to begin with. \textbf{$\{r_t\}$ matter, too}
  \end{frame}

\end{document}